\section{Важные упоминания о методе наименьших квадратов}
Здесь я не буду писать то же, что написано в любом учебнике или пособии, содержащем главу про МНК, но укажу на некоторые, может, не очевидные моменты во время разбора этой темы.

Первый, казалось бы простой, но не для всех очевидный момент, это то, что сам смысл метода кроется в его названии. Действительно, по определению вектор параметров функции $f(x, \vec \alpha)$ ищется из минимизации выражения:
\begin{equation}
    S(\alpha_1, \ldots, \alpha_k) = S(\vec \alpha) = \sum \limits_{i = 1}^{n}w_i\left[y_i - f(x, \vec \alpha)\right]^2
\end{equation}