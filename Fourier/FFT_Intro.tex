\section{Введение}
Пусть имеется действительная на интервале $\left[-\pi \leqslant t \leqslant \pi \right]$, конченая и непрерывная функция $f(t)$. Тогда на этом интервале ее можно представить в виде ряда Фурье:
\begin{equation}
    f(t) = A_0 + \sum \limits_{k = 1}^{\infty}\left[A_k \cos{kt} + B-k \sin{kt} \right]
    \label{Fourier_series}
\end{equation}
Коэффициенты ряда можно найти по формулам:
\begin{align}
    A_0 & {} = \frac1{2\pi}\int \limits^{\pi}_{-\pi}f(t)\,dt \\
    A_k & {} = \frac1{\pi} \int \limits^{\pi}_{-\pi}f(t) \cos{kt}\, dt \\
    B_k & {} = \frac1{\pi} \int \limits^{\pi}_{-\pi}f(t) \sin{kt}\, dt
    \label{Fourier coofitients}
\end{align}

Откуда же берутся эти формулы?

В данном случае функции $\varphi_k(t) = \sin{kt}$ и $\psi_k(t) = \cos{kt}$ и $\varphi_0(t) = 1$ для любых целых неотрицательных $k$ это базис, на котором строится целевая функция $f(t)$. Для таких функций определено скалярное произведение:
\begin{equation}
    \langle g(t) \text{,} h(t) \rangle = \int \limits^{\pi}_{-\pi} g(t)h(t)\, dt
\end{equation}
Причем:
\begin{align*}
    \langle \varphi_k, \varphi_m \rangle & {} = 0 & k \neq m \\
    \langle \psi_k, \psi_m \rangle & {} = 0 & k \neq m\\
    \langle \psi_k, \varphi_m \rangle & {} = 0 &  \forall k, m \in\mathbbm{N} \cup 0
\end{align*}
Однако при этом:
\begin{equation}
    \langle \varphi_k, \varphi_k \rangle = \langle \psi_k, \psi_k \rangle = \pi \qquad \forall k \in\mathbbm{N} \cup 0\\
\end{equation}
Отсюда, если мы скалярно умножим формулу (\ref{Fourier_series}) по обе стороны знака равенства на $\varphi_k(t)$, $\psi_k(t)$ и $\varphi_0$, то получим формулы (\ref{Fourier coofitients}).
    
    Зная, что гармонические функции $\sin{x}$ и $\cos{x}$ представимы через формулы Эйлера:
\begin{equation}
    \sin{x} = \frac{e^{ix} - e^{-ix}}{2i}\qquad   \cos{x} = \frac{e^{ix} + e^{-ix}}{2}
\end{equation}
Поэтому можем представить ряд Фурье в обобщенном виде:
\begin{equation}
    f(t) = \sum \limits^{+\infty}_{k = -\infty} C_k e^{-ikt}
\end{equation}
Где:
\begin{equation}
    C_k = \frac1{2\pi}\int \limits_{-\pi}^{\pi}f(t)e^{-ikt}\,dt
    \label{ob_ck}
\end{equation}
Связь между (\ref{Fourier coofitients}) и (\ref{ob_ck}) выражается формулами:
\begin{equation*}
    C_k=A_k - iB_k \qquad C_{-k}= A_k + iB_k
\end{equation*}

Однако разложение можно обобщить еще дальше, а именно представить, что частоты $k$ изменяются не дискретно, а непрерывны. В таком виде интегралы:
\begin{align}
    f(t) & {} = \int \limits^{+\infty}_{-\infty}F(\omega)e^{2\pi i \omega t}\, d\omega\\
    F(\omega) & {}= \int \limits^{+\infty}_{-\infty}f(\tau)e^{-2\pi i \omega \tau}\, d\tau
\end{align}
называются взаимообратными трансформантами Фурье.

Вообще в реальном мире сигнал с детектора (импульс), можно представить в виде гистограммы. На самом деле так и происходит, ведь мы не можем записать аналоговый сигнал в файл, однако можем записывать значение напряжения или тока через малые промежутки времени, как это делают цифровые осциллографы и диджитайзеры. Так что по факту, когда мы получаем данные для обработки, он всегда является гистограммой, а для них применяется другое - дискретное преобразование Фурье.

Дискретное преобразование Фурье в качестве базиса имеет функции на строго определенных частотах, определяемых исключительно количеством проведенных за период измерений сигнала.
Формула дискретного преобразования Фурье выглядит следующим образом:
\begin{equation}
    \mathrm{X}(j) = \sum \limits_{k = 0}^{N-1}A_k\exp{-\frac{2\pi i }{N}jk}
\end{equation}
Здесь $j$ - дискретное время. $j = 0\text{, }1\text{,}\ldots\text{,}N-1$.
$k$ - индекс дискретной частоты. $k = 0\text{, }1\text{,}\ldots\text{,}N-1$. Сама дискретная частота определяется формулой $\frac{k}{T}$, где $T$ - период в течении которого проводились измерения. $A_k$ - комплексные коэффициенты, фактически комплексные амплитуды сигналов, то есть хранящие информацию и об амплитуде и о фазе сигнала. $N$ - количество измеренных точек (сэмплов). Для расчета Фурье-образа или оригинала требуется $N^2$ операций комплексного сложения и умножения, однако если выкинуть члены ряда, значения которых мы значем, то вычисления будут проходить быстрее. На основе этого факта Кули и Тьюки придумали алгоритм быстрого преобразования Фурье. Пусть массив, на который мы накладываем преобразование Фурье имеет размер $N = 2^m$. Тогда выражение $N\log_2 N = mN$ будет характеризовать время выполнения расчетов. Например при $m =13$ необходимо выполнить около 8 тыс. операций. Обычное Фурье преобразование справится с таким массивом за полчаса, а быстрое всего за 5 миллисекунд. 